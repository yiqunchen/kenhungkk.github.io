\documentclass{article}

\usepackage{graphicx}

\begin{document}

{\bf Quiz \#2; Tuesday, date: 01/23/2018}

{\bf MATH 53 Multivariable Calculus with Stankova}

{\bf Section \#114; time: 2 -- 3:30 pm}

{\bf GSI name: Kenneth Hung}

{\bf Student name: SOLUTIONS}

\vspace*{0.25in}

\begin{enumerate}
\item Find an equation of the sphere that passes through the origin and whose center is $(1, -2, 2)$.

{\em Solution.} The radius of the sphere is given by
\[
\sqrt{(1 - 0)^2 + (-2 - 0)^2 + (2 - 0)^2} = 3.
\]
Thus the equation of the sphere is
\[
(x - 1)^2 + (y + 2)^2 + (z - 2)^2 = 9.
\]

\item {\em True / False?} Suppose $f$ and $g$ are functions of $t$, then the following two parametric curves has the same tangent line at $t = 0$:
\[
x = f(t), ~~~~ y = g(t);
\]
and
\[
x = f(-2t), ~~~~ y = g(-2t).
\]

{\em Solution.} True. There are two ways to arrive at this conclusion. First, the two parametric curves is in fact the same Cartesian curve, and $t = 0$ corresponds to the same point in both parametric curve, so the tangent line must be the same.

Alternatively, the two tangent line has the same point on it (the point corresponding to $t = 0$). The slope of the tangent is the second parametric curve, by chain rule, is
\[
\left.\frac{-2g'(-2t)}{-2f'(-2t)}\right|_{t = 0} = \left.\frac{g'(-2t)}{f'(-2t)}\right|_{t = 0} = \frac{g'(0)}{f'(0)},
\]
the same as that in the first parametric curve.

\item {\em True / False?} Given a polar curve $r = f(\theta)$, the area under the curve and above the $x$-axis from $\theta = \alpha$ to $\theta = \beta$ is always given by
\[
\int_\alpha^\beta \frac{1}{2} f(\theta)^2 \,d\theta.
\]

{\em Solution.} False. The integral formula provided computes the area bounded by the rays $\theta = \alpha$ and $\theta = \beta$, rather than the area under the curve.

\end{enumerate}

\end{document}