\documentclass{article}

\usepackage{amsmath}

\begin{document}

{\bf Quiz \#3; Tuesday, date: 02/06/2018}

{\bf MATH 53 Multivariable Calculus with Stankova}

{\bf Section \#117; time: 5 -- 6:30 pm}

{\bf GSI name: Kenneth Hung}

{\bf Student name:}

\vspace*{0.25in}

\begin{enumerate}
\item Find a nonzero vector orthogonal to the plane through the points $P$, $Q$, and $R$; find the area of triangle $PQR$.
\[
P(-6, 2, 4), ~~~~ Q(-8, 2, 6), ~~~~ R(-9, 4, 3).
\]

\item {\em True / False?} For any vectors $\mathbf{a}$, $\mathbf{b}$ and $\mathbf{c}$,
\[
(\mathbf{a} \cdot \mathbf{b}) \mathbf{c} \text{ and } (\mathbf{a} \cdot \mathbf{c}) \mathbf{b}
\]
may not be parallel but will always have the same magnitude by associativity.

\item {\em True / False?} Suppose $A$ and $B$ are two planes that intersect at a line $\ell$. To find the angle between $A$ and $B$, we can follow the recipe here:
\begin{itemize}
\item first, select a point $C$ on line $\ell$;
\item then, select lines $\ell_A$ and $\ell_B$ through $C$ and orthogonal to $\ell$, on $A$ and $B$;
\item finally, find the angle between lines $\ell_A$ and $\ell_B$.
\end{itemize}
\end{enumerate}

\end{document}