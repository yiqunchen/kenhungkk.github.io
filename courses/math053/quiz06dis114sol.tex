\documentclass{article}

\usepackage{amsmath,amsfonts}

\newcommand{\rr}{\mathbf{r}}

\begin{document}

{\bf Quiz \#6; Tuesday, date: 02/27/2018}

{\bf MATH 53 Multivariable Calculus with Stankova}

{\bf Section \#114; time: 2 -- 3:30 pm}

{\bf GSI name: Kenneth Hung}

{\bf Student name: SOLUTIONS}

\vspace*{0.25in}

\begin{enumerate}
\item Find the limit, if it exists, or show that the limit does not exist.
\[
\lim_{(x, y) \to (0, 0)} \frac{xy^3}{x^2 + y^6}
\]

{\em Solution.} On the $x$-axis, $f(x, 0) = 0$ for $x \ne 0$, so $f(x, y) \to 0$ as $(x, y) \to (0, 0)$ along the $x$-axis. Approaching $(0, 0)$ along the curve $x = y^3$ gives $f(y^3, y) = y^6 / 2y^6 = \frac{1}{2}$ for $y \ne 0$, so along this path $f(x, y) \to \frac{1}{2}$ as $(x, y) \to (0, 0)$. Thus the limit does not exist.

\item {\em True / False?} If $f$ is a function whose domain contains points arbitrarily close to $(2, 3)$, then
\[
\lim_{(x, y) \to (2, 3)} f(x, y) = (2, 3).
\]

{\em Solution.} {\bf False.} From the definition of continuity, the above is only true if the function is continuous at $(2, 3)$. In general functions may not be continuous at $(2, 3)$, e.g.
\[
f(x, y) = \begin{cases}
1 & \text{if } (x, y) = (2, 3) \\
0 & \text{otherwise}
\end{cases}
\]

\item {\em True / False?} Consider two functions $f$ and $g$ that are both defined on the domain of $f$.  Suppose the domain of $f$, $D_f$ is contained in the domain of $g$, $D_g$ (i.e.\ $D_f$ is a subset of $D_g$) and $f(x) = g(x)$ for any points $x$ in $D_f$. If the origin is in $D_f$ and $f$ is continuous at the origin, then $g$ is also continuous at the origin.

{\em Solution.} {\bf False.} Even the function is the same, expanding the domain can make a continuous function no longer continuous. Consider the example
\begin{align*}
f(x, y) & = 0 \text{ with domain } \{(x, y): x = 0\} \\
g(x, y) & = \begin{cases}
0 & \text{if } x = 0 \\
1 & \text{if } x = 1
\end{cases} \text{ with domain } \mathbb{R}^2
\end{align*}
The two functions are equal in the domain of $f$ and the domain of $g$ contains the domain of $g$. The values of the function $g$ in the larger domain may not conform to the requirements of continuity. Hence it is possible for $f$ to be continuous at the origin with $g$ not continuous at the origin.
\end{enumerate}

\end{document}