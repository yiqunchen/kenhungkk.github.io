\documentclass{article}

\usepackage{amsmath}

\newcommand{\qq}{\mathbf{q}}
\newcommand{\rr}{\mathbf{r}}
\newcommand{\ii}{\mathbf{i}}
\newcommand{\jj}{\mathbf{j}}
\newcommand{\kk}{\mathbf{k}}

\begin{document}

{\bf Quiz \#5; Tuesday, date: 02/20/2018}

{\bf MATH 53 Multivariable Calculus with Stankova}

{\bf Section \#117; time: 5 -- 6:30 pm}

{\bf GSI name: Kenneth Hung}

{\bf Student name: SOLUTIONS}

\vspace*{0.25in}

\begin{enumerate}
\item Find the tangential and normal components of the acceleration vector.
\[
\rr(t) = t \ii + 4e^{t/2} \jj + 2e^t \kk
\]

{\em Solution.} We want to apply the formula for the two components. We start by computing the derivatives.
\begin{align*}
\rr'(t) & = \ii + 2e^{t/2} \jj + 2e^t \kk, \\
\rr''(t) & = e^{t/2} \jj + 2e^t \kk.
\end{align*}

By Formula 9 and 10 on pg.\ 875,
\begin{align*}
a_T & = \frac{\rr'(t) \cdot \rr''(t)}{|\rr'(t)|} = \frac{2e^t + 4e^{2t}}{\sqrt{1 + 4e^t + 4e^{2t}}} = 2e^t. \\
a_N & = \frac{|\rr'(t) \times \rr''(t)|}{|\rr'(t)|} = \frac{|2e^{3t/2} \ii - 2e^t \jj + e^{t/2}|}{1 + 2e^t} = e^{t/2}.
\end{align*}

\item {\em True / False?} Suppose the curve $\rr(t)$ goes through the origin. A new curve formed by shrinking the curve $\rr(t)$ towards the origin by a factor of $2$. (In other words, a point $\mathbf{v}$ is shrunk to $\mathbf{v} / 2$.) The curvature at the origin is doubled.

{\em Solution.} {\bf True.} Suppose the new curve is parametrized by $\qq(t) = \rr(t) / 2$. The curvature of $\qq$ is given by
\[
\kappa_{\qq}(t) = \frac{|\qq'(t) \times \qq''(t)|}{|\qq'(t)|^3} = 2 \frac{|\rr'(t) \times \rr''(t)|}{|\rr'(t)|^3} = 2 \kappa_{\rr}(t).
\]
In particular, the curvature at the origin is doubled.

\item {\em True / False?} For a smooth space curve $\rr(t)$ that is on the $x, y$-plane, the binormal vector (when defined) must either be $\kk$ for all $t$ or $-\kk$ for all $t$, depending on which way the curve is traversed.

{\em Solution.} {\bf False.} While the binormal vector must either be $\kk$, $-\kk$, the sign may switch whenever the curvature is zero. In other words, for a curve with both counterclockwise and clockwise part, the binormal vector can be $\kk$ some time and $-\kk$ in some other time.
\end{enumerate}

\end{document}