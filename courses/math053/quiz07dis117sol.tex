\documentclass{article}

\usepackage{amsmath}

\newcommand{\rr}{\mathbf{r}}
\newcommand{\ii}{\mathbf{i}}
\newcommand{\jj}{\mathbf{j}}
\newcommand{\kk}{\mathbf{k}}

\begin{document}

{\bf Quiz \#7; Tuesday, date: 03/06/2018}

{\bf MATH 53 Multivariable Calculus with Stankova}

{\bf Section \#117; time: 5 -- 6:30 pm}

{\bf GSI name: Kenneth Hung}

{\bf Student name:}

\vspace*{0.25in}

\begin{enumerate}
\item Find the first partial derivatives of $f(x, y) = x^y$. Then use chain rule to find the derivative of $t^t$ with respect to $t$.

{\em Solution.} We find the partial derivatives first. We have
\[
f_x = y x^{y-1}, ~~~~ f_y = x^y \ln x.
\]
Now we consider when $x = t$ and $y = t$, so the derivative of $t^t$ is
\[
f_x \frac{dx}{dt} + f_y \frac{dy}{dt} = yx^{y-1} + x^y \ln x = t t^{t-1} + t^t \ln t = t^t (1 + \ln t).
\]

\item {\em True / False?} Given a function at $f$, defined on a disc near the origin. To show that $f$ is not differentiable at the origin, it suffices to find three curves through the origin, such that their tangent lines at the origin do not lie on the same plane.

{\em Solution.} {\bf True.} If $f$ is differentiable at the origin, then it must have a tangent plane (the linearization) that contains the tangent lines to any curves through the origin, contradicting the assumption that the three tangent lines do not lie on the same plane.

\item {\em True / False?} Suppose
\[
z = f(x, y), ~~~~ x = g(t, u), ~~~~ y = h(u, v),
\]
then by chain rule we have
\[
\frac{\partial z}{\partial t} = \frac{\partial z}{\partial x} \frac{\partial x}{\partial t}.
\]

{\em Solution.} {\bf True.} Since $y$ does not depend on $t$, the partial derivative $\partial y / \partial t = 0$ and so chain rule reduces to the above.
\end{enumerate}

\end{document}