\documentclass{article}

\begin{document}

{\bf Quiz \#1; Tuesday, date: 01/23/2018}

{\bf MATH 53 Multivariable Calculus with Stankova}

{\bf Section \#114; time: 2 -- 3:30 pm}

{\bf GSI name: Kenneth Hung}

{\bf Student name:}

\vspace*{0.25in}

\begin{enumerate}
\item Consider the parametric equation for a curve:
\[
x = \sqrt{t - 1}, ~~~~ y = \sqrt{t + 3}.
\]
Eliminate the parameter to find a Cartesian equation of the curve. Sketch the curve and indicate with an arrow the direction in which the curve is traced as the parameter increases.

\item {\em True / False?} A Cartesian equation $f(x, y) = 0$ of a curve in the plane can always be re-written to define the curve by some function: $y = g(x)$, or by some function: $x = h(y)$.

\item {\em True / False?} The polar curve
\[
r = 2 \cos \theta, ~~~~ 0 \le \theta \le 6 \pi
\]
is a circle centered at $(1, 0)$, traversed $6$ times.
\end{enumerate}

\end{document}