\documentclass{article}

\usepackage{amsmath,amsfonts,tikz}

\newcommand{\ii}{\mathbf{i}}
\newcommand{\jj}{\mathbf{j}}
\newcommand{\kk}{\mathbf{k}}
\newcommand{\rr}{\mathbf{r}}
\newcommand{\vv}{\mathbf{v}}
\newcommand{\aaa}{\mathbf{a}}
\newcommand{\FF}{\mathbf{F}}
\newcommand{\TT}{\mathbf{T}}
\newcommand{\NN}{\mathbf{N}}
\newcommand{\BB}{\mathbf{B}}
\newcommand{\SSS}{\mathbf{S}}

\DeclareMathOperator*{\curl}{curl}

\begin{document}

{\bf Worksheet \#28; date: 05/01/2018}

{\bf MATH 53 Multivariable Calculus}

\begin{enumerate}

\item {\em (Stewart 16.Rev.38)} Let
\[
\FF(x, y) = \frac{(2x^3 + 2xy^2 - 2y) \ii + (2y^3 + 2x^2 y + 2x) \jj}{x^2 + y^2}.
\]
Evaluate $\oint_C \FF \cdot d\rr$, where $C$ is a curve looped around the origin once counterclockwise.

\item {\em (Stewart 16.7.43)} A fluid has density $870 \text{ kg}/\text{m}^3$ and flows with velocity $\vv = z \ii + y^2 \jj + x^2 \kk$, where $x$, $y$, and $z$ are measured in meters and the components of $\vv$ in meters per second. Find the rate of flow outward through the cylinder $x^2 + y^2 = 4$, $0 \le z \le 1$.

\item {\em True / False?} If a vector field $\vec{F}(x,y,z)$ is conservative on an open region in $\mathbb{R}^3$ that contains an oriented, smooth surface $S$ with a simple, closed, smooth boundary curve $C$, then even if we orient $C$ negatively, both sides of ST will yield $0$.

\item {\em True / False?} When we change the variables in a double integral using a transformation $T: x=g(u,v), y=h(u,v)$ that sends the original domain $D\subset \mathbb{R}^2_{x,y}$ to $S\subset \mathbb{R}^2_{u,v}$, we need to ensure that $T$ is one-to-one but not necessarily onto.

\item {\em True / False?} When trying to calculate/find the limit $\displaystyle{\lim_{(x,y)\rightarrow(a,b)}f(x,y)=L}$, we insist on $(x,y){\not =} (a,b)$ because the limit may exist at $(a,b)$ even if the function is not defined or discontinuous there.

\item {\em True / False?} By changing the $x,y$−-coordinate system, we can see that the graph of function $f(x,y)=xy$ is actually a parabolic hyperboloid.
\end{enumerate}

\end{document}