\documentclass{article}

\begin{document}

{\bf Worksheet \#2; date: 01/23/2018}

{\bf MATH 53 Multivariable Calculus}

\begin{enumerate}
\item {\em (Stewart 10.1.33)} Find parametric equations for the path of a particle that moves along the circle $x^2 + (y - 1)^2 = 4$ in the manner described.
\begin{enumerate}
\item Once around clockwise, starting at $(2, 1)$
\item Three times around counterclockwise, start at $(2, 1)$
\item Halfway around counterclockwise, starting at $(0, 3)$
\end{enumerate}

\item {\em (Stewart 10.3.5)} Consider the points $(-4, 4)$ and $(3, 3 \sqrt{3})$, given in Cartesian coordinates.
\begin{enumerate}
\item Find the polar coordinate $(r, \theta)$ of the point, where $r > 0$ and $0 \le \theta < 2\pi$.
\item Find the polar coordinate $(r, \theta)$ of the point, where $r < 0$ and $0 \le \theta < 2\pi$
\end{enumerate}

\item {\em (Stewart 10.3.12)} Sketch the region in the plane consisting of points whose polar coordinates satisfy the given conditions.
\[
r \ge 1, ~~~~ \pi \le \theta \le 2\pi
\]

\item {\em (Stewart 10.3.23)} Find a polar equation for the curve represented by the given Cartesian equation.
\[
y = 1 + 3x.
\]

\item {\em (Stewart 10.3.27; modified)} Find both the Cartesian equation and the polar equation for each of the following described curves. Decide for yourself which one would you rather do / use.
\begin{enumerate}
\item A line through the origin that makes an angle of $\pi / 6$ with the positive axis.
\item A vertical line through the point $(3, 3)$.
\end{enumerate}

\item {\em (Challenging; Stewart 10.3.52; modified)} Rewrite the curve $(x^2 + y^2)^2 = 4 x^2 y^2$ in polar coordinates. Sketch the curve afterwards.

\item {\em (Stewart 10.2.3)} Find an equation of the tangent to the curve at the point corresponding to the given value of the parameter.
\[
x = t^3 + 1, ~~~~ y = t^4 + t; ~~~~ t = -1.
\]

\item {\em (Stewart 10.2.7)} Find an equation of the tangent to the curve at the given point by two methods: (a) without eliminating the parameter and (b) by first eliminating the parameter.
\[
x = 1 + \ln t, ~~~~ y = t^2 + 2; ~~~~ (1, 3).
\]

\item {\em (Stewart 10.2.16)} Find $dy / dx$ and $d^2y / dx^2$. For which values of $t$ is the curve concave upward?
\[
x = \cos t, ~~~~ y = \sin 2t, ~~~~ 0 < t < \pi.
\]

\item {\em (Stewart 10.2.25)} Show that the curve $x = \cos t$, $y = \sin t \cos t$ has two tangents at $(0, 0)$ and find their equations. Sketch the curve.

\end{enumerate}

\end{document}