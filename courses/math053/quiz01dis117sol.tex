\documentclass{article}

\usepackage{amsmath}

\begin{document}

{\bf Quiz \#1; Tuesday, date: 01/23/2018}

{\bf MATH 53 Multivariable Calculus with Stankova}

{\bf Section \#117; time: 5 -- 6:30 pm}

{\bf GSI name: Kenneth Hung}

{\bf Student name: SOLUTIONS}

\vspace*{0.25in}

\begin{enumerate}
\item Identify the curve
\[
r = 6 \sec \theta
\]
by finding a Cartesian equation for the curve.

\vspace*{0.25in}

{\em Solution.} Use the equations defining polar coordinates to convert the polar equation to a Cartesian equation. In this question it is possible to reach the solution by substituting the polar representation of $r$ into either the $x$ or $y$ equation.

(Solution through $x$)
\begin{align*}
x & = r \cos \theta \\
x & = 6 \sec \theta \cos \theta \\
x & = 6.
\end{align*}

(Solution through $y$)
\begin{align*}
y & = r \sin \theta \\
y & = 6 \sec \theta \sin \theta \\
y & = \frac{6}{\cos \theta} \sin \theta \\
y & = 6 \tan \theta \\
y & = 6 \frac{y}{x} \\
x & = 4.
\end{align*}

The curve is a vertical line $x = 6$.

\item {\em True / False?} Given a curve in parametric form
\[
x = f(t), ~~~~ y = g(t), ~~~~ -\infty < t < \infty.
\]
This is always the same curve as
\[
x = f(s^3), ~~~~ y = g(s^3), ~~~~ -\infty < s < \infty.
\]

{\em Solution.} {\bf True}; for any point $(f(t), g(t))$ on the first curve, it can be thought of as $(f(s^3), g(s^3))$ if we take $s = \sqrt[3]{s}$. On the other hand, for any point $(f(s^3), g(s^3))$ on the first curve, it can be thought of as $(f(t), g(t))$ if we take $t = s^3$.

\item {\em True / False?} All points can be described uniquely using polar coordinates $(r, \theta)$, once we require $r \ge 0$ and $0 \le \theta < 2\pi$.

{\em Solution.} {\bf False}; the origin is always a bit weird, in that it cannot be uniquely described as $(0, \theta)$ for any $\theta$ is still the origin.

{\em Solution.}
\end{enumerate}

\end{document}