\documentclass{article}

\usepackage{amsmath,amsfonts,tikz}

\newcommand{\ii}{\mathbf{i}}
\newcommand{\jj}{\mathbf{j}}
\newcommand{\kk}{\mathbf{k}}
\newcommand{\rr}{\mathbf{r}}
\newcommand{\vv}{\mathbf{v}}
\newcommand{\aaa}{\mathbf{a}}
\newcommand{\TT}{\mathbf{T}}
\newcommand{\NN}{\mathbf{N}}
\newcommand{\BB}{\mathbf{B}}

\begin{document}

{\bf Worksheet \#17; date: 03/15/2018}

{\bf MATH 53 Multivariable Calculus}

\begin{enumerate}
\item {\em (Stewart 14.8.44)} Find the maximum and minimum volumes of a rectanglar box whole surface area is $1500~\text{cm}^2$ and whose total edge length is $200~\text{cm}$.

\item {\em (Stewart 14.8.45)} The plane $x + y + 2z = 2$ intersects the paraboloid $z = x^2 + y^2$ in an ellipse. Find the points on this ellipse that are nearest to and farther from the origin.

\item {\em (Teaser for 14.8.49)}
\begin{enumerate}
\item Find the minimum value of
\[
f(x_1, x_2, x_3, x_4, x_5) = \sqrt{\frac{x_1^2 + x_2^2 + x_3^2 + x_4^2 + x_5^2}{5}}
\]
given that $x_1, x_2, x_3, x_4, x_5$ are positive numbers and $x_1 + x_2 + x_3 + x_4 + x_5 = c$, where $c$ is a constant.

\item Deduce from part (a) that if $x_1, x_2, x_3, x_4, x_5$ are positive numbers, then
\[
\sqrt{\frac{x_1^2 + x_2^2 + x_3^2 + x_4^2 + x_5^2}{5}} \ge \frac{x_1 + x_2 + x_3 + x_4 + x_5}{5}.
\]
This inequality says that the root-mean-square of $5$ numbers is no larger than the arithmetic mean of the numbers. Under what circumstances are these two means equal?
\end{enumerate}

\item Evaluate the double integral by first identifying it as the volume of a solid.
\[
\iint_R (3 - y) \,dA, ~~~~ R = [0, 1] \times [0, 1].
\]

\item The integral $\iint_R \cos y \,dA$, where $R = [0, 3] \times [0, \pi/2]$, represents the volume of a solid. Sketch the solid and find its volume.

\end{enumerate}

\end{document}