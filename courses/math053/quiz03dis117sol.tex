\documentclass{article}

\usepackage{amsmath}

\begin{document}

{\bf Quiz \#3; Tuesday, date: 02/06/2018}

{\bf MATH 53 Multivariable Calculus with Stankova}

{\bf Section \#117; time: 5 -- 6:30 pm}

{\bf GSI name: Kenneth Hung}

{\bf Student name: SOLUTIONS}

\vspace*{0.25in}

\begin{enumerate}
\item Find a nonzero vector orthogonal to the plane through the points $P$, $Q$, and $R$; find the area of triangle $PQR$.
\[
P(-6, 2, 4), ~~~~ Q(-8, 2, 6), ~~~~ R(-9, 4, 3).
\]

{\em Solution.} We start by finding two vectors representing the two edges of the triangle first, say, $\overrightarrow{PQ}$ and $\overrightarrow{PR}$.
\begin{align*}
\overrightarrow{PQ} & = \langle -2, 0, 2\rangle \\
\overrightarrow{PR} & = \langle -3, 2, -1\rangle
\end{align*}
Now we take the cross product to find a vector orthogonal to the plane.
\begin{align*}
\overrightarrow{PQ} \times \overrightarrow{PR} & = \begin{vmatrix}
\mathbf{i} & \mathbf{j} & \mathbf{k} \\
-2 & 0 & 2 \\
-3 & 2 & -1
\end{vmatrix} \\
& = \left\langle \begin{vmatrix}
0 & 2 \\ 2 & -1
\end{vmatrix}, -\begin{vmatrix}
-2 & 2 \\ -3 & -1
\end{vmatrix}, \begin{vmatrix}
-2 & 0 \\ -3 & 2
\end{vmatrix}\right\rangle \\
& = \langle -4, 8, -4\rangle.
\end{align*}
The area of the triangle is given by
\begin{align*}
\frac{1}{2} |\overrightarrow{PQ}| \cdot |\overrightarrow{PR}| \cdot \sin \angle QPR & = \frac{1}{2} |\overrightarrow{PQ} \times \overrightarrow{PR}| \\
& = \frac{1}{2} \sqrt{(-4)^2 + 8^2 + (-4)^2} \\
& = 2 \sqrt{6}.
\end{align*}

\item {\em True / False?} For any vectors $\mathbf{a}$, $\mathbf{b}$ and $\mathbf{c}$,
\[
(\mathbf{a} \cdot \mathbf{b}) \mathbf{c} \text{ and } (\mathbf{a} \cdot \mathbf{c}) \mathbf{b}
\]
may not be parallel but will always have the same magnitude by associativity.

{\em Solution.} False. They generally do not have the same magnitude.
\[
|(\mathbf{a} \cdot \mathbf{b}) \mathbf{c}| = |\mathbf{a}| \cdot |\mathbf{b})| \cdot |\mathbf{c}| \cos \theta_{\mathbf{a}, \mathbf{b}},
\]
while
\[
|(\mathbf{a} \cdot \mathbf{c}) \mathbf{b}| = |\mathbf{a}| \cdot |\mathbf{b})| \cdot |\mathbf{c}| \cos \theta_{\mathbf{a}, \mathbf{c}}.
\]
Here $\theta_{\mathbf{a}, \mathbf{b}}$ is the angle between $\mathbf{a}$ and $\mathbf{b}$. Similarly $\theta_{\mathbf{a}, \mathbf{c}}$ is defined. Since these two angles do not need to be the same, the magnitudes above do not need to be the same. There is no associativity with both scalar multiplication and dot product in the mix.

\item {\em True / False?} Suppose $A$ and $B$ are two planes that intersect at a line $\ell$. To find the angle between $A$ and $B$, we can follow the recipe here:
\begin{itemize}
\item first, select a point $C$ on line $\ell$;
\item then, select lines $\ell_A$ and $\ell_B$ through $C$ and orthogonal to $\ell$, on $A$ and $B$;
\item finally, find the angle between lines $\ell_A$ and $\ell_B$.
\end{itemize}

{\em Solution.} The plane containing $\ell_A$ and the normal vector to $A$ is orthogonal to $\ell$ and passes through $C$. Similarly, the plane containing $\ell_B$ and the normal vector to $B$ is orthogonal to $\ell$ and passes through $C$. In other words, $\ell_A$, $\ell_B$, and the normal vectors to $A$ and $B$ are all on the same plane. Since $\ell_A$ and the normal vector to $A$ are orthogonal, and $\ell_B$ and the normal vector to $B$, the angle between $\ell_A$ and $\ell_B$ is the same as that between the two normal vectors.
\end{enumerate}

\end{document}