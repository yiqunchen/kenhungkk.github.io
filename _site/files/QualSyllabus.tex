\documentclass{article}

\begin{document}

\title{Qualifying exam syllabus}
\author{Kenneth Hung}
\date{\today}
\maketitle

\begin{center}
{\sc Committee}

David Aldous, Steve Evans, William Fithian (Stat) , James Demmel (Chair)
\end{center}

\vspace{0.2in}

\begin{center}
{\sc Major topic: Probability theory (Probability)}
\end{center}

\begin{description}
\item[Preliminaries] $\sigma$-algebras, Dynkin's $\pi$-$\lambda$ theorem, Independence, Borel--Cantelli lemma, Kolmogorov's $0$-$1$ law, Kolmogorov's maximal inequality, Strong and weak laws of large numbers

\item[Central limit theorems] Weak convergence, Uniform integrability, Characteristic functions, I.I.D.\ central limit theorem, Lindeberg--Feller CLT

\item[Conditioning] Conditional probability, Conditional expectation, Conditional independence, Regular conditional probabilities

\item[Martinagles] Stopping times, Upcrossing inequality, A.S.\ convergence, Doob's decomposition, Doob's inequality, $L^p$ convergence, $L^1$ convergence, Reverse martingale convergence, Optional stopping theorem, Wald's identity

\item[Markov chains] Countable state space, Stationary measures, Convergence theorems, Recurrence and transience, Asymptotic behavior

\item[References] Durrett, Probability: Theory and Examples, Chapters 1--3, 5--6
\end{description}

\vspace{0.2in}

\begin{center}
{\sc Major topic: Theoretical statistics (Probability)}
\end{center}

\begin{description}
\item[Exponential family] Densities, Parameters, Moments, Cumulants, Generating functions

\item[Estimators and statistics] Risks, Estimators, Sufficient statistic, Complete statistic, Ancillary statistic, Factorization theorem, Minimal sufficient statistic, Basu's theorem, Rao--Blackwell theorem

\item[Unbiased estimation] UMVU estimators, Variance bounds, Fisher information, Cram\'er--Rao bound, Higher dimension variance bounds

\item[Bayesian estimation] Prior and posterior distribution, conjugate distribution

\item[Large sample theory] Convergence in probability, Convergence in distribution, Central limit theorem, Delta method, Asymptotic relative efficiency

\item[Estimating equations] Weak law for random functions, Kullback--Leibler divergence, Consistency of maximum likelihood estimator, limiting distribution for MLE, Confidence interval, asymptotic confidence interval

\item[Empirical Bayes] Empirical Bayes estimator, James--Stein estimator

\item[Hypothesis testing] Test function, Power, Significance, Neyman--Pearson lemma, Uniformly most powerful tests, Monotone likelihood ratio, Duality between testing and interval estimation, Generalized Neyman--Pearson lemma, Two-sided hypothesis, Unbiased test

\item[References] Keener, Theoretical Statistics, Chapters 2--4, 6--9, 11
\end{description}

\vspace{0.2in}

\begin{center}
{\sc Minor topic: Numerical linear algebra (Applied Mathematics)}
\end{center}

\begin{description}
\item[Linear equation solving] Gaussian elimination, Perturbation theory, Blocking algorithms, Special linear systems (real symmetric position definite matrices, symmetric indefinite matrices, general sparse matrices)

\item[Linear least squares] Normal equations, QR decomposition, Singular value decomposition, Householder transformations, Givens rotations, Rank-deficient least square problems, Perturbation theory

\item[Nonsymmetric eigenproblem] Power method, Inverse iteration, Orthogonal iteration, QR iteration, Tridiagonal and bidiagonal reduction, Perturbation theory

\item[Symmetric eigenproblem] Tridiagonal QR iteration, Rayleigh quotient iteration, Divide-and-conquer, Bisection and inverse iteration, Perturbation theory

\item[References] Demmel, Applied Linear Algebra, Chapters 2--5
\end{description}

\end{document}